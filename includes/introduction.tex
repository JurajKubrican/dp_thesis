Petriho siete sa využívajú na modelovanie paralelných a udalostných systémov. Biznis procesy, ktoré bežne prebiehajú v súčasných firmách sú jedným prípadom takýchto systémov. Pomocou rôznych rozšírení Petriho sietí sme schopný tak podrobne namodelovať biznis proces, že na základe modelu sme schopný vygenerovať funkčné systémy, ktoré dokážu tieto biznis procesy obsluhovať.  

Modelovací jazyk Petriflow stavia na niektorých známych rozšíreniach Petriho sietí a pridáva vlastnú funkcionalitu a formálny zápis, ktorý nám poskytuje dostatočný základ na to aby sme vedeli generovať komplexné procesne orientované systémy. 

V našej práci sa budeme venovať návrhu a implementácií jednej časti takéhoto systému. Časť ktorú budeme navrhovať je aplikačné rozhranie pre procesný server. ktoré umožní používateľom interagovať s procesným systémom. V našej práci sa budeme hlavne venovať práci s prechodmi, rolami a  dátovými poľami. 