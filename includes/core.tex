 

  

\section{Analýza problému} 

  

\subsection{Úvod do petriho sietí} 


\subsection{Procesný server} 

 Majme procesný server tento server v sebe obsahuje informáciu o rôznych procesoch. Tieto procesy majú rôznu štruktúru a sú v rôznych stavoch. Štruktúru týchto procesov môžeme reprezentovať pomocou Petriho sietí a ich stav pomocou rôznych značkovaní v týchto sieťach.  

% TODO - najprv odstavec pred tým 

  

\subsection{Aplikačné rozhranie} 

 Na to aby sme mohli k dátam pristupovať zvonka uzavretej domény procesného servera potrebujeme vytvoriť aplikačné rozhranie ktoré bude riešiť autentifikáciu, autorizáciu a dokumentáciu.  

% TODO 

  

  

  

  

\section{Špecifikácia} 

 V tejto kapitole najprv stručne opíšeme hlavnú funkcionalitu navrhovanej aplikácie, potom zadefinujeme funkcionálne a nefunkcionálne požiadavky na aplikáciu. 

  

Softvér, ktorý sme sa rozhodli implementovať bude slúžiť ako rozhranie medzi procesným serverom a internetom. Bude umožňovať klientovi získavať informácie o dátach v prechodoch petriho siete a bude mu umožňovať modifikovať stav siete (procesu) spúšťaním prechodov.  

  

\subsection{Funkcionálne požiadavky} 

\begin{enumerate} 

    \item Rozhranie bude umožňovať registrovať používateľov a priraďovať im roly 

    \item Umožní prihlásenie používateľa pomocou štandardného autentifikačného protokolu. 

    \item Autentifikovaným používateľom umožní prístup k dátam z tých prechodov ktoré majú právo čítať podľa ich roly. 

    \item Autentifikovaným používateľom umožní spúšťať prechody ktoré majú právo spúšťať podľa ich roly. 

    \item Pri spúšťaní prechodu prebehne validácia vstupných dát. V prípade nevalidných alebo nekompletných dát nepovolí spustenie prechodu. 

    \item Rozhranie poskytne online dokumentáciu prechodov v sieti, táto dokumentácia bude zahŕňať URL prechodu, potrebné dátové polia na spustenie prechodu a roly, ktoré sú oprávnené prechody spúšťať. 

    \item Rozhranie poskytne aplikačné rozhranie viacerým sieťam s rôznou štruktúrou. 

\end{enumerate}     

  

  

  

\subsection{Nefunkcionálne požiadavky} 

\begin{enumerate} 

    \item Rozhranie bude škálovateľné 

    \item Rozhranie bude napísané vo frameworku Spring Boot 

\end{enumerate} 

  

\section{Návrh} 

 V tejto kapitole najprv prejdeme architektúru nášho softvér 

% TODO - what have we done in this chapter? 

  

\subsection{Architektúra} 

 Aby sme splnili požiadavku na jednoduché škálovanie aplikácie a pre sprehľadnenie architektúry zvolili sme si architektúru mikroservisov. Táto architektúra pozostáva z viacerých oddelených častí, každá z týchto častí má svoju jasne definovanú funkciu. Takéto mikroservisy sú jednoducho testovateľné, dajú sa nasadzovať postupne a nezávisle od seba a softvér navrhnutý v tejto architektúre býva spravidla robustný a vysoko škálovateľný.  

  

Táto architektúra však nie je vhodná na menšie projekty, lebo réžia vzniknutého softvéru býva spravidla vysoká, lebo si vyžaduje viacero spustených inštancií servisov. Tiež nie je vhodná v prípadoch, kde sa čakávajú väčšie zmeny biznis logiky aplikácie. Pri väčšej smene biznis logiky je často nutné prerábať viacero servisov a zmena protokolu, ktorým medzi sebou komunikujú. 

  

  

\section{Implementácia} 

% TODO 

  

\subsection{Použité technológie} 

  

\subsubsection{Kotlin} 

 Kotlin je relatívne nový programovací jazyk, projekt Kotlin bol po prvý krát zverejnený v roku 2011 spoločnosťou JetBrains(Andrey Breslav). Bol vyvinutý ako moderný staticky typovaný jazyk, ktorý podporuje rýchlu kompiláciu do javy. V roku 2017 vyhlásil Google podporu pre Kotlin v operačnom systéme Android.  

Medzi jeho hlavné výhody patrí menší boilerplate (menej zbytočného kódu),  

Vylepšený systém typovania premenných. premenné môžu byť null a kompilátor vie odvodiť v ktorých prípadoch premenná null obsahovať môže a v ktorých nie,  

  

Jednoduchosť prechodu z Javy na kotlin štruktúra kódu je podobná jave, Jetbrains dokonca poskytuje transpiler, ktorý dokáže kód z Javy vo väčšine prípadov trasnpilovať do Kotlinu 

% TODO 

  

\subsubsection{Gradle} 

Gradle je voľne šíriteľný nástroj na automatizáciu zostavovania softvéru. Je stavaný na to aby bol schopný postaviť takmer ľubovoľný program. Podporuje jazyky ako java, C++ Python, a mnoho ďalších. Kotlin beží na JVM, takže sa dá používať na všetkých bežných platformách. V našom projekte sa gradle stará o stiahnutie závislostí nášho projektu, kompiláciu kódu a spustenie samotného skompilovaného programu. Konfigurácia nástroja prebieha pomocou konfiguračného súboru napísaného v jazyku Groovy, tieto konfiguračné  súbory sa v našom prípade použitia ukázali ako veľmi prehľadné a ľahké na použitie. Gradle taktiež používa pokročilú techniku memoizácie procesu stavby softvéru takže jeho výkon je pri opakovanej kompilácií vyšší. Alternatívny nástroj na ktorý sme zvažovali je Maven, no vybrali sme si Gradle, hlavne kvôli jeho lepšiemu výkonu a prehľadnejšej konfigurácií. 

  

%TODO - JVM to glossary 

  

  

\subsubsection{Spring boot} 

 Spring Boot je voľne šíriteľný framework založený na Jave. Je vyvíjaný a udržiavaný tímom Pivotal. Je určený na vytváranie nezávislých, produkčných aplikácií a mikro-služieb.  

% TODO - spring boot  

Pri práci so Spring boot budeme využívať návrhové vzory: 

  

Dependency injection / Inversion of control (Tým že v jednoduchých triedach pridáme anotáciu, vieme z frameworku zdediť nielen funkcie, ale aj control flow)  

  

Singleton (aplikácia vie zaručiť že z daného objektu sa v rámci jednej inštancie aplikácie vytvorí len jedna inštancia, ku ktorej sa dá pristupovať z celej aplikácie) 

  

Factory (aplikácia používa na vytváranie objektov tzv Beanov návrhový vzor factory) 

  

  

  

\subsubsection{Spring cloud} 

 Spring Cloud je framework, ktorý obsahuje bohatú sadu nástrojov na vytváranie mikroslužieb a cloudových riešení. Medzi nástroje Spring Clopudu patí:  

\begin{itemize} 

\item Cloud config - nástroj na distribúciu konfiguračných súborov medzi kontajnermi mikrosluzieb 

\item Service discovery - nástroj na registráciu a monitorovanie mikroservisov 

\item Gateway - Nástroj na routovanie a load balancing v rámci mikroservisov 

\item Cloud Authentication - Nástroj na riešenie komplexnej autentizácie a autorizácie v rámci 

\end{itemize} 

  

  

  

\subsection{Inštalácia a konfigurácia kontajnerov} 

 Všetky Spring Boot kontajnery sme inštalovali pomocou spring initializr 

% https://start.spring.io/ - link  

tento nástroj vygeneruje zip súbor so založeným projektom vo frameworku Spring Boot. Pri vytváraní projektu je možné si vybrať Jazyk v ktorom bude projekt založený a nástroj ktorý bude projekt zostavovať % --  

\subsubsection{Inštalácia a konfigurácia kontajnerov} 

  

  

  

  

\section{Testovanie} 

% TODO 

  
